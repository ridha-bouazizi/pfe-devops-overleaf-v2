\thispagestyle{empty}
\thispagestyle{empty}

\\
\vspace{2cm}
\begin{center}

\ungaramond{\textbf{\LARGE{Absract}}}
\end{center}
\vspace{0.8cm}
During my seven and a half month engineering studies internship at Systnaps, I successfully implemented a complete Platform-as-a-Service (PaaS) environment in the cloud using the Agile DevOps methodology. 

The project involved putting in place networking services, authentication and authorization services, distributed storage, CICD tools, DevOps pipelines, and a resilient disaster recovery strategy. Additionally, standard DevOps practices were implemented to ensure smooth and efficient collaboration between development and operations teams. 

The project was completed successfully, enabling Systnaps to rapidly develop and deploy applications with minimal downtime and improved quality. The implementation of a complete PaaS environment and the adoption of DevOps practices demonstrated the benefits of the cloud technology and the Agile DevOps methodology in improving the software development and delivery process. 

At the end of my internship, I was offered a permanent position at Systnaps as the head of the cloud and DevOps branch. This opportunity reflects the value that my contribution brought to the company and validates my skills and expertise in the field of cloud computing and DevOps. 

Overall, my internship at Systnaps was a great success, enabling me to gain a practical experience in cloud computing and DevOps and to make a valuable contribution to the company. 

% Keywords : 

\begin{flushright}
\textbf Ridha BOUAZIZI
\end{flushright}

\newpage
\thispagestyle{empty}
\begin{center}
\ungaramond{\textbf{\LARGE{Resumé}}}
\end{center}
\vspace{0.8cm}
Pendant mon stage d'ingénierie de sept mois et demi chez Systnaps, j'ai réussi à mettre en place un environnement complet de Plateforme en tant que service (PaaS) dans le cloud en utilisant la méthodologie Agile DevOps. 

Le projet consistait à mettre en place des services de mise en réseau, des services d'authentification et d'autorisation, du stockage distribué, des outils CICD, des pipelines DevOps et une stratégie résiliente de “reprise après sinistre”. 

De plus, les bonnes pratiques DevOps ont été mises en place pour assurer une collaboration fluide et efficace entre les équipes de développement et d'exploitation. 

Le projet a été mené à bien, permettant à Systnaps de développer et de déployer rapidement des applications avec un temps d'arrêt minimal et une qualité améliorée. 

La mise en place d'un environnement complet PaaS et l'adoption des pratiques DevOps ont démontré les avantages de la technologie cloud et de la méthodologie Agile DevOps pour améliorer le processus de développement et de livraison de logiciels. 

À la fin de mon stage, j'ai été offert un poste permanent chez Systnaps en tant que responsable de la branche cloud et DevOps. Cette opportunité reflète la valeur que ma contribution a apportée à l'entreprise et valide mes compétences et mon expertise dans le domaine de l'informatique en nuage et de DevOps. 

Dans l'ensemble, mon stage chez Systnaps a été un grand succès, me permettant d'acquérir une expérience pratique en informatique en nuage et en DevOps et de faire une contribution précieuse à l'entreprise.

% Mots clés : 

\begin{flushright}
\textbf Ridha BOUAZIZI
\end{flushright}