\renewcommand{\thepage}{}
\setcounter{listing}{0}
\pagestyle{empty}
\renewcommand{\thispagestyle}[1]{}
\clearpage
\renewcommand{\thelisting}{}
\fancyhead[R]{\ungaramond\small\textbf{}}

\addtocontents{toc}{\cftpagenumbersoff{chapter}}
\addcontentsline{toc}{chapter}{Annex A: ClusterIssuer for TLS certificates.}\textcolor{white}{I} 
% \begin{center}
\ungaramond{\textbf{\LARGE{Annex A :\hspace{7mm}ClusterIssuer for TLS certificates.}}}
% \end{center}
\label{ann:annexeA}

\begin{listing}[H]
    \inputminted{yaml}{codeListing/cert_manager_cluster_issuer.yml}
    \caption[]{Sample ClusterIssuer using DNS01 challenge.}
    \label{lst:annexe1}
\end{listing}


\newpage
\pagestyle{empty}
\renewcommand{\thispagestyle}[1]{}
\clearpage
\addtocontents{toc}{\cftpagenumbersoff{chapter}}
\addcontentsline{toc}{chapter}{Annex B: CRD middlewares for authentication.}\textcolor{white}{I} 
\ungaramond{\textbf{\LARGE{Annex B :\hspace{7mm}CRD middlewares for authentication.}}}
% \end{center}
\label{ann:annexeB}

\begin{listing}[H]
    \inputminted{yaml}{codeListing/middleware_forward_auth.yml}
    \caption[]{The "forward-auth" middleware CRD.}
    \label{lst:annexe1}
\end{listing}

\begin{listing}[H]
    \inputminted{yaml}{codeListing/middleware_headers.yml}
    \caption[]{The "headers" middleware CRD.}
    \label{lst:annexe1}
\end{listing}

\newpage
\pagestyle{empty}
\renewcommand{\thispagestyle}[1]{}
\clearpage
\addtocontents{toc}{\cftpagenumbersoff{chapter}}
\addcontentsline{toc}{chapter}{Annex C: Sample ingressroute declaration.}\textcolor{white}{I}
\ungaramond{\textbf{\LARGE{Annex C :\hspace{7mm}Sample ingressroute declaration.}}}
% \end{center}
\label{ann:annexeC}

\begin{listing}[H]
    \inputminted{yaml}{codeListing/authelia_ingressroute.yml}
    \caption[]{A sample ingressroute CRD with the "forward-auth" middleware chain.}
    \label{lst:annexe1}
\end{listing}

\newpage
\pagestyle{empty}
\renewcommand{\thispagestyle}[1]{}
\clearpage
\addtocontents{toc}{\cftpagenumbersoff{chapter}}
\addcontentsline{toc}{chapter}{Annex D: CronJob for backing up DB data.}\textcolor{white}{I}
\ungaramond{\textbf{\LARGE{Annex D :\hspace{7mm}CronJob for backing up DB data.}}}
% \end{center}
\label{ann:annexeC}

\newenvironment{code}{\captionsetup{type=listing}}{}
\begin{code}
\begin{minted}[linenos, breaklines, firstnumber=last]{yaml}
apiVersion: batch/v1
kind: CronJob
metadata:
  creationTimestamp: null
  name: mongodump-demodb
  namespace: mongodb
spec:
  schedule: "0 0 4,12,20,28 * *"
  concurrencyPolicy: Forbid
  failedJobsHistoryLimit: 2
  successfulJobsHistoryLimit: 3
  startingDeadlineSeconds: 60
  suspend: false
  jobTemplate:
    metadata:
    spec:
      template:
        metadata:
        spec:
          containers:
          - command:
            - python3
            - /root/dumper/dumper.py
            env:
            - name: AWS_ACCESS_KEY_ID
              valueFrom:
                secretKeyRef:
                  key: AWS_ACCESS_KEY_ID
                  name: aws-s3-secret-gra
            - name: AWS_SECRET_ACCESS_KEY
              valueFrom:
                secretKeyRef:
                  key: AWS_SECRET_ACCESS_KEY
                  name: aws-s3-secret-gra
            - name: AWS_DEFAULT_REGION
              valueFrom:
                secretKeyRef:
                  key: AWS_DEFAULT_REGION
                  name: aws-s3-secret-gra
            - name: MONGODB_ROOT_PASSWORD
              valueFrom:
                secretKeyRef:
                  key: mongodb-root-password
                  name: mongodb
            - name: MONGODB_CONFIG
              value: /root/config/mongo-config
            - name: DB_NAME
              value: "demodb"
            - name: TEAMS_WEBHOOK_URL
              valueFrom:
                secretKeyRef:
                  key: TEAMS_WEBHOOK_URL
                  name: teams-webhook-url
            image: core-harbor.huxium.io/library/mongodb-backup-restore:latest
            imagePullPolicy: Always
            name: mongodump-backup
            resources: 
              limits:
                cpu: 1000m
                memory: 512Mi
              requests:
                cpu: 100m
                memory: 128Mi
            volumeMounts:
            - name: config
              mountPath: "/root/dumper/"
            - name: aws
              mountPath: "/.aws"
            - name: mongo-config
              mountPath: "/root/config/"
            - name: backup
              mountPath: "/opt/backup"
          dnsPolicy: ClusterFirst
          imagePullSecrets:
          - name: regcred
          restartPolicy: Never
          schedulerName: default-scheduler
          volumes:
          - name: aws
            configMap:
              name: backup-cm
              defaultMode: 420
              items:
              - key: credentials
                path: credentials
              - key: config
                path: config
          - name: mongo-config
            secret:
              secretName: mongo-config
              defaultMode: 420
              items:
              - key: mongo-config
                path: mongo-config
          - name: config
            configMap:
              name: backup-cm
              defaultMode: 420
              items:
              - key: dump.py
                path: dump.py
              - key: requirements.txt
                path: requirements.txt
          - name: backup
            persistentVolumeClaim:
              claimName: mongodb-backup
\end{minted}
\caption[]{CronJob for backing up mongodb structured data periodically.}
\end{code}
